\documentclass[a4paper]{article}
\usepackage[dutch]{babel}
\usepackage{mathtools}
\usepackage{graphicx}
\usepackage{setspace}
\graphicspath{ {Afbeeldingen/}}
\title{Practicum Numerieke Wiskunde}
\author{Barbara Ameloot, Tobias Baert}
\date{Mei 2018}
\begin{document}
\maketitle
Bewijzen 1.1:
\\{}
\\Om een element  $l_{i,j}$ af te leiden gebruikten we in 1.1 steeds een vergelijking van de vorm:
\\{}
\\$a_{ij} = \sum_k^{i-1} {l_{ik}\cdot u_{kj}}$ met $ i < j$ 
\\{}
\\Daaruit leiden we af dat:
\\ \textbullet{ } $l_{i,1}$, het eerste element van elke rij van L, gelijk is aan nul als en slechts als $a_{i,j}$ gelijk is aan nul.
\\ \textbullet { } $ l_{i,j}$ gelijk is aan 0 als en slechts als $a_{i,j}$ gelijk is aan nul EN alle elementen $l_{i,k}$ met $k = 1, ..., i - 1$ (dit zijn de elementen links in dezelfde rij) gelijk zijn aan 0.
\\{}
\\Als gevolg daarvan wordt elke rij van links naar rechts met nullen gevuld tot men aankomt op de subdiagonaal. Omdat L verder een benedendriehoeksmatrix is, is L dus tridiagonaal.
\\{}
\\Om een element $u_{i,j}$ af te leiden gebruikten we in 1.1 steeds een vergelijking van de vorm:
\\{}
\\$a_{ij} = \sum_k^{i-1} {l_{ik}\cdot u_{kj}} + u_{ij}$ met$ i >= j $
\\{}
\\Daaruit leiden we af dat:
\\ \textbullet{ } $u_{1,j}$, het eerste element van elke kolom van$ U$, gelijk is aan nul als en slechts als $a_{i,j}$ gelijk is aan 0.
\\ \textbullet{ } $u_{i,j}$ gelijk is aan nul als en slechts als $a_{i,j}$ gelijk is aan nul EN alle elementen $u_{k,j}$ met $k = 1, ..., i - 1$ (dit zijn de elementen erboven in dezelfde kolom) gelijk zijn aan 0.
\\{}
\\Als gevolg daarvan wordt elke kolom van boven naar onderen met nullen gevuld tot men aankomt op de superdiagonaal. Omdat $U$ verder een bovendriehoeksmatrix is, is $U$ dus tridiagonaal.
\end{document}
