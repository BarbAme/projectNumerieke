\documentclass[a4paper]{article}
\usepackage[dutch]{babel}
\usepackage{mathtools}
\title{Practicum Numerieke Wiskunde}
\author{Barbara Ameloot, Tobias Baert}
\date{Mei 2018}
\begin{document}
\maketitle
   \section{Opdracht 1.1}
Het matrixproduct  $A=L_{E}\cdot U$ :
\\
\[
 \begin{bmatrix}
  a_{1,1} & a_{1,2} & \cdots & a_{1,n} \\
  a_{2,1} & a_{2,2} & \cdots & a_{2,n} \\
  \vdots  & \vdots  & \ddots & \vdots  \\
  a_{n,1} & a_{n,2} & \cdots & a_{n,n} 
 \end{bmatrix}
=
\begin{bmatrix}
 1 & 0 & \cdots & 0 \\
  l_{2,1} &1 & \cdots &0 \\
  \vdots  & \vdots  & \ddots & \vdots  \\
  l_{n,1} &l_{n,2} & \cdots & 1 
\end{bmatrix}
\cdot
\begin{bmatrix}
  u_{1,1} & u_{1,2} & \cdots & u_{1,n} \\
  0 & u_{2,2} & \cdots & u_{2,n} \\
  \vdots  & \vdots  & \ddots & \vdots  \\
 0 & 0 & \cdots & u_{n,n}
\end{bmatrix}
\]
\\
\begin{minipage}[c]{\textwidth}
%\begin{samepage} %elementen op zelfde pagina houden
Als we de elementen van dit matrixproduct elementgewijs uitschrijven krijgen we:
\begin{tabbing} %voor layout
\\$a_{1,1} = 1 \cdot u_{1,1}$\hspace{6.0cm} \= $\rightarrow  u_{1,1}$
\\$a_{1,2} = 1 \cdot u_{1,2}$ \> $\rightarrow  u_{1,2}$
\\$a_{1,3} = 1 \cdot u_{1,3}$ \> $\rightarrow  u_{1,3}$
\\ {...}
\\$a_{1,n} = 1 \cdot u_{1,n}$ \> $\rightarrow  u_{1,n}$
\\ {}
\\$a_{2,1} = l_{2,1} \cdot u_{1,1}$ \> $\rightarrow l_{2,1}$
\\$a_{2,2} = l_{2,1} \cdot u_{1,2} + 1 \cdot u_{2,2}$ \> $\rightarrow u_{2,2}$
\\$a_{2,3} = l_{2,1} \cdot u_{1,3} + 1 \cdot u_{2,3}$ \> $\rightarrow u_{2,3}$
\\{...}
\\$a_{2,n} = l_{2,1} \cdot u_{1,n} + 1 \cdot u_{2,n}$ \> $\rightarrow u_{2,n}$
\\{}
\\$a_{3,1} = l_{3,1} \cdot u_{1,1}$ \> $\rightarrow l_{3,1}$
\\$a_{3,2} = l_{3,1} \cdot u_{1,2} +  l_{3,2} \cdot u_{2,2}$ \> $\rightarrow l_{3,2}$
\\$a_{3,3} = l_{3,1} \cdot u_{1,3} +  l_{3,2} \cdot u_{2,3} + 1 \cdot u_{3,3}$ \> $\rightarrow u_{3,3}$
\\{...}
\\$a_{3,n} = l_{3,1} \cdot u_{1,n} +  l_{3,2} \cdot u_{2,n} + 1 \cdot u_{3,n}$ \> $\rightarrow u_{3,n}$
\\{}
\\$a_{4,1} = l_{4,1} \cdot u_{1,1}$ \> $\rightarrow l_{4,1}$
\\$a_{4,2} = l_{4,1} \cdot u_{1,2} +  l_{4,2} \cdot u_{2,2}$ \> $\rightarrow l_{4,2}$
\\$a_{4,3} = l_{4,1} \cdot u_{1,3} +  l_{4,2} \cdot u_{2,3} + l_{4,3} \cdot u_{3,3}$ \> $\rightarrow l_{4,3}$
\\$a_{4,4} = l_{4,1} \cdot u_{1,4} +  l_{4,2} \cdot u_{2,4} + l_{4,3} \cdot u_{3,4} + 1 \cdot u_{4,4}$ \> $\rightarrow u_{4,4}$
\\{...}
\\$a_{4,n} = l_{4,1} \cdot u_{1,n} +  l_{4,2} \cdot u_{2,n} + l_{4,3} \cdot u_{3,n} + 1 \cdot u_{4,n}$ \> $\rightarrow u_{4,n}$
\\{}
\end{tabbing}
\end{minipage}
\pagebreak
%Elementen berekenen
\begin{verse}
\begin{tabbing}
We kunnen de elementen $u_{i,j}$ en $ l_{i,j}$ van $U$ en $L_{E}$ als volgt berekenen:
\\Voor $i = j \Rightarrow$  \= 
\\ \>$u_{i,j} =  a_{i,j}$
\\ \>$l_{i,j} =  1$
\\{}
\\Voor $i < j \Rightarrow$
\\ \>$u_{i,j} =  a_{i,j}$
\\ \>$l_{i,j} =  0$
\\{}
\\Voor $i > j  \Rightarrow$ 
\\ \>$u_{i,j} =  0$
\\ \>$l_{i,j} =  a_{i,j}$
\end{tabbing}
\end{verse}
\section{Opdracht 1.2}
\end{document}

